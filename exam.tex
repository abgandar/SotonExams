%
% General advice: to update the counts in the paper, it needs to be compiled at least twice (but you knew that)!
%
%\documentclass[]{uosexam}
% Options that can be provided to the class, for example:
\documentclass[answers]{uosexam}
%
% draft - Add the word "Draft" in the background, show layout errors
% supplementary - This is a supplementary exam (changes wording in front matter as per faculty guidance)
% main - This is a "normal" exam (default)
% answers - Include the answers in the output
% no-answers - Only output the exam (default)
% astrobooklet - Astro equations booklet is provided
% no-astrobooklet - No AStro equations booklet is provided (default)
% calvertfarrar - Calvert Farrar engineering data book is provided (default)
% no-calvertfarrar - Calvert Farrar is not provided
% dictionary - Word-to-word dictionary is allowed (default)
% no-dictionary - Don't allow dictionary
% calculator - Allow uni-approved calculators (default)
% no-calculator - Don't allow calculators
% hours - given duration is in hours
% no-hours - given duration is in minutes (default)

\begin{document}
\unitTitle{Introduction to \LaTeX}
\authors{Alexander Wittig}
\unitcode{SESA6666}
\semester{2}
\examyear{2018}        % NOTE: this has been renamed from earlier versions of the template to not clash with built-in \year commands in packages
\duration{120} 		% minutes unless "hours" given as class option
%\QtoA{9}	                   % number of questions to answer (default: all)
\specialMarkingScheme{   % This will be added at the end of the marking scheme instructions on the front matter page
Read all questions carefully before starting to answer any of them.
}
\specialProvided{        % This will be added at the end of the list of provided boxes on the front matter page
\isprovided{A \LaTeX\ suit in your size is provided}        % isprovided{} draws a nice box around the text and centers it
}
\specialInstructions{    % This will be added at the end of the instructions on the front matter page
You may place one phone call to Donald Knuth during the exam.
}

% By default, the information in the front matter is automatically calculated.
%
% If your exam uses Sections AND you have complicated instructions, you need to manually
% overwrite the blurb for the front page as the \QtoA command doesn't handle sections:
%\NumberOfQuestions{This paper contains NINETY-NINE sections with TWO questions each. \par
%Answer all ODD NUMBERED question in EVERY OTHER section.}
%
% You can also change the automatic marking scheme text using
% \MarkingScheme{An outline marking scheme is shown in brackets to the right of each question. This exam is marked out of \textbf{69 marks}.}

\maketitle

%%%%%%%%%%%%%%%%%%%%%%%%%%%%%%%%%%%%%%%%%%%%%%%%%%%

\begin{luacode*}
	require('Lua')
	muS = 1.327e11			-- km^3/s^2
	mu = 3.986e5			-- km^3/s^2
	RE = 6371				-- km
	au = 1.496e9			-- km
	G = 6.67430e-11			-- m^3/kg s^2
\end{luacode*}
\newcommand{\lp}[1]{\luadirect{print(#1)}}

%%%%%%%%%%%%%%%%%%%%%%%%%%%%%%%%%%%%%%%%%%%%%%%%%%%

% If you use sections, you can start a new section like this:
%\section

%  Start a question worth 20 marks.
\begin{question}{20}
What is the capital of Australia?

\begin{answer}
% Anything written in here is only output if the answers option is set
Canberra
\end{answer}
\end{question}

%%%%%%%%%%%%%%%%%%%%%

\begin{question}{15}
This question is made up of several parts:

% Start a list of the parts
\begin{parts}
    \part[5]{      % start a new part of the question worth 5 marks
    How many parts does this question have? Note also the information in \tref{tab:info} and \tref{tab:mytable}.

    \begin{table}[htp]
    \caption{Very important data!}
    \begin{center}
    \begin{tabular}{c|cc}
       & A & B \\
    \hline 
    a & \SI{1}{\m} & \SI{2.6e3}{\kg\km} \\
    b & \num{34669} & \si{\N\m}
    \end{tabular}
    \end{center}
    \label{tab:info}
    \end{table}
    
    \begin{answer}
    $$p=3$$
    \end{answer}
    }

    \part[5]{
    Which part of this question is the best?
    \begin{itemize}
    \item Part (i)
    \item Part (ii)
    \item Part (iii)
    \item Part (iv)
    \end{itemize}
    
    \begin{answer}
    Clearly it is part (ii).

% If you're using LuaTeX, you can calculate the answer to your questions conveniently in Lua
\ifluatex
    % A simple vector and matrix class are provided as demonstrated in this piece of code.
    \begin{luacode*}
        mu = 1.327e11
        au = 1.496e8
        r = Vector:new( {1.3369, 0.66845, 0} )*au
        v = Vector:new( {0, 25, 0} )
        
        h = r:cross(v)
        e = v:cross(h)/mu - r/r:norm()
        
        a = math.rad(70)
        R = Matrix:new({{  math.cos(a), math.sin(a), 0 },
                        { -math.sin(a), math.cos(a), 0 },
                        {  0,           0,           0 }})
        R[3][3] = 1     -- fix the bottom right entry
        e_rot = R*e
    \end{luacode*}

    % Variables from Lua code can be printed directly into the document as shown here.
    % An optional argument specifying the number of digits after the decimal point maybe specified (default: 4)
    Also, for $\mu=\directlua{print(mu)}$ the eccentricity vector is:
    $$\vec{e} = \frac{\vec{v}\times\vec{h}}{\mu}-\frac{\vec{r}}{r} = \directlua{e:print(6)}$$
    After applying rotation matrix $$R=\directlua{print(R)}$$ it becomes $$\vec{e}_{rot}=R\cdot \vec{e}=\directlua{e_rot:print()}.$$
\fi
% End of LuaTeX code
    \end{answer}
    }

% If you want to insert manual page breaks before a part (e.g. to prevent a page break within a part), use the \pagebreak command (not \newpage or \clearpage) like this
\pagebreak

    \part[5]{
    How often do the letters \emph{e} and \emph{n} appear in this part?\\

    \begin{answer}
    % the \pmark and \pmarks{#} commands can be used to indicate partial marks for steps in the answer
    e appears 6 times \pmarks{4} \\
    n appears 4 times \pmark
    \end{answer}
    }
\end{parts}
\end{question}

%%%%%%%%%%%%%%%%%%%%%

\begin{question}{10}
This is a really poorly written question. The marks don't even add up!

\begin{parts}
    \part[5]{
    How many marks is this part of the question worth?

    \begin{answer}
    5 marks \pmarks{4}
    % If partial marks are present in the answer but don't add up, you get a warning in the LaTeX output 
    % and the mark for the part is shown in red (only if answers option is set)
    \end{answer}
    }

    \part[10]{
    What's the total number of marks for this question (see also \fref{fig:myfigure})?
    
    \begin{figure}[htb]
    \begin{center}
    Your figure could appear here!
    %\includegraphics[width=0.7\textwidth]{figs/myfigure.png}
    \caption{Placeholder for a pretty figure.}
    \label{fig:myfigure}
    \end{center}
    \end{figure}

    \begin{answer}
    10 marks.
    % If the total number of marks at the end of a question does not add up to what was declared, you get
    % a warning in the LaTeX output and the total mark is shown in red.
    \end{answer}
    }
\end{parts}
\end{question}

%%%%%%%%%%%%%%%%%%%%%%%%%%%%%%%%%%%%%%%%%%%%%%%%%%%

\begin{formulasheet}
A formula sheet is optional. If this environment is present, a note is automatically included in the front matter
using (almost) the wording from the faculty guidance.

If you don't want a formula sheet, just remove this environment.

Formula sheets must always be the last thing in the exam. Note how the footer on each page is adjusted
in accordance with faculty guidelines (END OF PAPER after last question, etc.)

\begin{table}[htb]
\caption{Some really cool data could be shown here so the students can enjoy the exam along with this long caption.}
\begin{center}
\begin{tabular}{|c|c|}
\hline
But & it \\
\hline
is & not \\
\hline
\end{tabular}
\end{center}
\label{tab:mytable}
\end{table}

\end{formulasheet}

\end{document}
